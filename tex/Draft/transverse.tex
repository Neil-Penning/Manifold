\section{Transverse}

I learned about Transverse from \cite{lee_2013}

\begin{boxDefinition}{Transverse}
Let \( A \) and \( B \) be submanifolds of \( M^n \).

Let \( p \in A \) and \( p \in B \).
We say that \( A \) and \( B \) intersect transversely at \( p \)

Note that \( T_pA \) and \( T_pB \) are subspaces of \( T_pM \).
Let \( \mathcal A \) and \( \mathcal B \) be bases of \( T_pA \) and \(T_pB \) respectively,
    we call say that \( A \) and \( B \) intersect transversely if \( \mathcal A \cup \mathcal B \) is a basis for \( T_pM \).
\end{boxDefinition}
\begin{boxExample}{Not Transverse}
The two submanifolds
\begin{align*} 
    S_1^1 &:= \{ (x,y) : (x-2)^2 + y^2 = 2^2 \}
    \\
    S_2^1 &:= \{ (x,y) : (x+2)^2 + y^2 = 2^2 \}
\end{align*}
of \( \RR^2 \) intersect at \( (0, 0) \) non-transversely because 
\begin{align*} 
    T_{(0,0)}S_1^1 = \span \left\{ \vtwo 0{-1} \right\}
    T_{(0,0)}S_2^1 = \span \left\{ \vtwo 0{-1} \right\}
\end{align*}
Which are the same dimension 1 subspace of \( T_{(0,0)}\RR^2 \).
\end{boxExample}
\begin{boxExample}{Transverse}
Consider the \( xy \) plane and the \( z \) axis as submanifolds of \( \RR^3 \).
\end{boxExample}
