\chapter{One}

The Real Line \index{Real Line}, \( \RR \) is the unique complete ordered field.

Using the structure of \( \RR \),
    we may define Connectedness, Convergence, Distance, Continuity, Differentiability, and Integration.

But \( \RR \) is not the only object we may define these (and more) constructs on.

This brings us to the motivating definition of this book:
\begin{boxIntuition}{Manifold}
\begin{boxTODO}{}
\end{boxTODO}

But what structure are we relying on?
\begin{itemize}
    \item All Cauchy Sequences converge
    \item Well Ordering
    \item Archimedean Principle
    \item Limits are Unique
    \item Metric Space
    \item Countable Basis
\end{itemize}

The actual definition includes unmotivated phraseology such as "locally euclidean", "second countable", and "Hausdorff's Unique Limit Space".
\end{boxIntuition}
We may do calculus on subsets of \( \RR \), for instance 
\begin{boxExample}{Some Simple Manifolds}
\begin{align*} 
    [0, 1] &:= \{ x \in \RR\ :\ 0 \leq x \leq 1 \}\\
    (3, 5) &:= \{ x \in \RR\ :\ 3 < x < 5 \}\\
    [0, 1] &:= \{ x \in \RR\ :\ 0 \leq x \leq 1 \}\\
\end{align*}
\end{boxExample}


\begin{boxTODO}{}
Vague:
The goal of this chapter is to ask what other \( \RR \)-like objects we may work with
    and then to classify them.
\end{boxTODO}


\section{Manifold Topology}
To begin, we must ask when two objects are "the same"
\begin{boxDefinition}{Open Set}
\end{boxDefinition}
\begin{boxDefinition}{Basis}
\end{boxDefinition}
\begin{boxDefinition}{Continuity}
\end{boxDefinition}
\begin{boxDefinition}{}
\end{boxDefinition}
\section{Doing Calculus}

\section{The Axioms}
The spaces that we have been dealing with, \( \RR \), \( S^1 \), The Ray, and the Interval, 
    all share a subtle properties.
To mathematicians,
    these properties are called Hausdorff's Condition and Second Countability.

We will construct =
\subsection{Euclid's Space}
\subsection{Hausdorff's Unique Limit Space}
% Introduces the Quotient Topology and creates the Bug-Eye-Line
When discussing open sets and "closeness",
    we have subtly assumed that two points may not be infinitely close together.
That is,
    in all the spaces we work with,
    we have assumed that for every two distinct points \( x \) and \( y \),
    there is an open set \( A \) that contains \( x \) and an open set \( B \) that contains \( y \) and \( A \) is disjoint from \( B \).
This property is known as the Hausdorff Axiom.
\begin{boxDefinition}{Hausdorff Axiom}
\end{boxDefinition}
Before we move on, convince yourself that \( R \) is indeed Hausdorff.
\begin{boxProblem}{Prove \( \RR \) is Hausdorff}
Prove that \( \RR \) satisfies the Hausdorff Axiom.
\end{boxProblem}

Now, in an act of poof form, we will 



\subsection{Alexandroff's Long Line}
