\chapter{Outline}

Poincare started topology in \cite{poincare_1895}


Milnore's Charactaristic Classes book \cite{milnor_stasheff_2009}.


\section{One}
The classification of One-manifolds is known and simple to work through.
\section{Two}
\subsection{Classification}
We follow the proof in \cite{munkres_2018} of classifying closed surfaces
In \cite{richards_1963} Ian Richards proves the classification theorem for non-compact surfaces.
\subsection{Surface Groups}
In \cite{dehn_1987}, Dehn proved the word problem for surface groups.

\section{Three}
\subsection{Classification}
\subsection{Foliations}

\section{Four}
\subsection{Intersection Form}
In \cite{whitehead_1949}, Whitehead introduced the intersection form (probably)

\subsection{Kirby}
To understand Four Manifolds, we need to understand Kirby Calculus \cite{gompf_stipsicz_1999}.

\section{Five}
Smale, in \cite{smale_1960} proved the Generalized Poincare Conjecture in Higher Dimensions.
In the following year, he classified all 5 manifolds in \cite{smale_1962}

\section{Six}
Hirsch proved that Piecewise linear and topological manifolds are not equivalent for dimension greater than 6 in \cite{hirsch_1975}

\section{Seven}
In 1956, Milnor proved that there are spheres that are homeomorphic, but not diffeomorphic in \cite{milnor_1956}
