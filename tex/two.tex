\chapter*{Two}


In Chapter One, we defined manifolds as an object we can do calculus on.

We did this by using a chart to locally inherit structure from \( \RR \),
    and an atlas of charts 


\begin{boxDefinition}{Dimension of a Manifold}
    If \( M \) is locally euclidean to \( \RR^n \), then the dimension of \( M \) is \( n \).
\end{boxDefinition}
\begin{boxNote}{Dimension Notation and Prononciation}
    The notation of a manifold reflects the dimension it is, not the dimension it is embedded in.
    For instance \( \RR^2 \) (read "R two") is two dimensional.
    And \( S^5 \) (read "five sphere") is five dimensional.
\end{boxNote}
\begin{boxTheorem}{}
\end{boxTheorem}

\section{To Do List for 2 dimensions}
\begin{boxTODO}{}
\begin{itemize}
\item \( \RR^2 \) exists
\item 2 manifolds are distinct from 1 manifolds
\item Pure manifolds (technically the disjoint union of \(R^2\) and \(R^1\) is a manifold)
\item The sphere
\item The torus
\item The Real Projective Plane
\item Constructing 2 manifolds via connect sums
\item Classifying closed 2 manifolds, using quotient topology
\item Intersection forms
\item Immersion Submersion
\item 2-Sphere Eversion
\item Covers and Universal Covers of manifolds - H2 covers tori
\begin{itemize}
    \item Fundamental Group
    \item Dehn's Word Problem on the Fundamental Group
\end{itemize}
\item Classifying all 3-manifolds https://math.stackexchange.com/questions/5588/classification-theorem-for-non-compact-2-manifolds-2-manifolds-with-boundary
\end{itemize}

    Mapping Class Groups

\end{boxTODO}
